\documentclass{article}
\usepackage{amsmath}
\usepackage{fullpage}

\begin{document}

\title{Variational Form Derivation for Navier-Stokes Equations}
\author{Brandon Denton}
\date{January 19, 2018}

\section{Steady State Compressible Continuity Equation - Variational Form Derivation}

The integral from of the continuity equation for compressible flow is given by:

\begin{equation} \label{eq1}
\int \frac{d\rho}{dt}  + \nabla \cdot (\rho \vec{u} ) dV = 0
\end{equation}

where $\rho$ is the density of the fluid, $\vec{u}$ is the velocity vector and $t$ is time.

Applying the assumption of steady state, $\frac{d\rho}{dt} = 0$, reduces the continuity equation to

\begin{equation} \label{eq2}
\int \nabla \cdot (\rho \vec{u} ) dV = 0
\end{equation}

The next step in formulating the variational form of the compresseible continuity equation is to 
rewrite the equation in is residual form. Since the continuity equation is already equal to zero,
it is already in is residual form:

\begin{equation} \label{residual}
r = \int \nabla \cdot (\rho \vec{u} ) dV = 0
\end{equation}

Now we must mulitply the residual by an admissible test function, $v$, which must be square integratable.
This results in:

\begin{equation} \label{residual-v}
rv = \int v \nabla \cdot (\rho \vec{u} ) dV = 0
\end{equation}

Equation \ref{residual-v} can be recast in a more useful form by using the vector identity

\begin{equation} \label{vecIdent}
\nabla \cdot (\Psi \vec{A}) = \Psi \nabla \cdot \vec{A} + \vec{A} \cdot \nabla\Psi
\end{equation}

and letting $\Psi = v$ and $\vec{A} = \rho \vec{u}$ results in

\begin{equation} \label{vecIdentapply}
\nabla \cdot (v \rho \vec{u}) = v \nabla \cdot (\rho \vec{u}) + \rho \vec{u} \cdot \nabla v
\end{equation}

rearranging equation \ref{vecIdentapply} gives

\begin{equation} \label{vecIdentappre}
v \nabla \cdot (\rho \vec{u}) = \nabla \cdot (v \rho \vec{u}) - \rho \vec{u} \cdot \nabla v
\end{equation}

Substituting equation \ref{vecIdentappre} into equation \ref{residual-v} gives

\begin{equation} \label{residual-sub1}
rv = \int \nabla \cdot (v \rho \vec{u}) dV - \int \rho \vec{u} \cdot \nabla v dV = 0
\end{equation}

rearranging equation \ref{residual-sub1} gives

\begin{equation} \label{residual-rearr}
\int \rho \vec{u} \cdot \nabla v dV = \int \nabla \cdot (v \rho \vec{u}) dV
\end{equation}

Upon further inspection, the Divergence theorem can be applied to the term on the right hand
side of equation \ref{residual-rearr} resulting in the final variational form of the continuity
equation.

\begin{equation}
\int \rho \vec{u} \cdot \nabla v dV = \int v \rho \vec{u} \cdot \hat{n} dS
\end{equation}

I have stopped at this variational form of the continuity equation for compressible flow as it 
affords us the ability to substitute different equation of state models for the density, $\rho$, 
which I believe is in line with the overall philosophy of GRINS and Petsc. 

\end{document}




